\mbox{}
\vspace{3cm}

\epigraph{\textit{%
Ce texte est la transcription d'un message donné par Chuck Smith (pasteur principal de l'église Calvary Chapel de
Costa Mesa en Californie) aux responsables de l'église Calvary Chapel de West Covina, le 13~décembre~1988.
}}{}

\lettrine{L}{a} philosophie du mouvement Calvary Chapel concernant le rôle et la fonction de l’Église, se trouve dans le passage
d'\ibibleverse{Eph}(4:8-13), où Paul déclare que Jésus-Christ est monté aux cieux, après être descendu dans les régions les
plus profondes de la terre. Quand il est monté dans les hauteurs, il a emmené des captifs et il a fait des dons aux
hommes\frcolon les uns comme apôtres, les autres comme prophètes, les autres comme évangélistes, les autres comme
pasteurs et enseignants. Paul explique ensuite pourquoi ces hommes ont été donnés\frcolon pour le perfectionnement des
saints, pour l’œuvre du service, pour l’édification du corps de Christ. Nous croyons que l’Église existe principalement
pour Jésus; pour Lui procurer du plaisir; pour que nous apportions louange et gloire à Sa grâce. Le Seigneur a créé
l’Église pour son bon plaisir, et ainsi, l’Église existe tout d’abord pour lui; c’est Son Église. Christ a dit\frcolon\og Sur ce rocher
je bâtirai Mon Église\fg{}. Je fais partie de Son Église. Une seule personne peut dire \og Mon Église\fg{} et c’est Jésus. C’est
Son Église. Ce qui est intéressant, c’est que vous ne pouvez pas en faire partie en vous y inscrivant. Il vous faut y
naître. Nous naissons de nouveau par l’Esprit de Dieu dans l’Église de Jésus-Christ. C’est Son Église.

Quelle est donc la raison d'être de Son Église ? C'est de glorifier Dieu; d’être l’instrument de Dieu pour servir le
Seigneur. Mais en deuxième lieu, l’Église existe pour l’édification c'est-à-dire pour le perfectionnement des saints, pour
amener les croyants à une pleine maturité leur permettant de s’engager dans l’œuvre de service.

Quand j’étais étudiant en Faculté de Théologie, Oswald J. Smith, le pasteur de l’Église du Peuple de Toronto au
Canada, église réputée dans le monde entier pour son dynamisme missionnaire, accordait une importance extrême
aux activités missionnaires à l’étranger. Dans les séminaires auxquels j’ai assisté, je l’ai entendu répéter et répéter
que la raison d'être principale de l’Église était l’évangélisation du monde. Je l’avais entendu dire cela si souvent que
j’avais fini par l’accepter comme une vérité d’Évangile. Aussi, quand j’ai commencé dans le ministère, j’ai cherché à
évangéliser le monde. Mes sermons avaient toujours pour but d'évangéliser les incroyants. Je les terminais toujours
par une invitation\frcolon\og Recueillez-vous, fermez les yeux sans tricher, et que ceux qui souhaitent recevoir Jésus-Christ ce
soir, lève la main et la rabaisse\fg{}. Tout était conçu en vue de l’évangélisation. Je cherchais à être un évangéliste parce
que j’estimais que la raison d'être de l’église était l’évangélisation du monde. C'est ce qu'on m'avait bien mis dans la
tête.

Très vite cependant, j'ai découvert que la chose la plus difficile au monde, c’est d’essayer d’être quelqu'un que Dieu
ne vous a pas destiné à être. Paul demandait\frcolon tous sont-ils apôtres, tous sont-ils prophètes, tous sont-ils évangélistes
? La réponse est, évidemment, \og non\fg{}. Tout le monde n’a pas reçu la vocation d’évangéliste. Tout le monde n’a pas
reçu la vocation d’un pasteur-enseignant. Tout le monde n’a pas reçu la vocation d’un prophète. Essayer d’être
quelqu'un d'autre que la personne créée par Dieu, est la chose la plus difficile au monde. J’essayais d’être quelque
chose que Dieu ne m’avait pas demandé d'être.

Paul commence sa lettre aux Ephésiens en disant\frcolon\og Paul, un apôtre par la volonté de Dieu\fg{}. C'est quelque chose que
je peux accepter. Je peux dire
\og Chuck, pasteur-enseignant par la volonté de Dieu\fg{}. Il est important de découvrir ce que nous sommes par la volonté
de Dieu. Pendant des années, je voulais être \og Chuck, évangéliste, par la volonté de Chuck\fg{}. Ce n’était pas par la
volonté de Dieu. J’essayais de me conformer au moule du mouvement dans lequel je servais. C’était une
dénomination qui mettait l'accent sur l’évangélisation. L’exhortation était plus appréciée que l’explication des textes
bibliques, aussi le rôle de pasteur enseignant n'était pas encouragé. On attendait de tous les pasteurs qu'ils soient des
évangélistes, et nous nous efforcions donc d’être des évangélistes. Ma tentative d'être un évangéliste se soldait par
un échec lamentable. Ma femme essayait de m’aider. Elle voyait combien j'étais frustré et elle me conseillait\frcolon\og Chéri,
tu n’es pas assez dynamique\fg{}. Elle me disait\frcolon\og Regarde Billy Graham, il ne reste pas immobile derrière sa chaire, il se
déplace sur l'estrade \fg{}. Elle ajoutait\frcolon\og Il va falloir que tu apprennes à te déplacer, à être plus dynamique\fg{}. J’essayais,
mais ça ne marchait pas. J’étais frustré parce que je cherchais à être quelqu'un d'autre que la personne créée par
Dieu.

M'étant mis à lire et à étudier la Parole de Dieu à la recherche du passage biblique qui disait que la raison d'être de
l'Église est l’évangélisation du monde, j'ai réalisé que je ne le trouvais pas; je ne peux toujours pas le trouver! Mais
j’ai trouvé dans \ibiblechvs{Eph}(4:), que Dieu a placé des hommes doués, des apôtres, des évangélistes, des prophètes, des
pasteurs enseignants, pour le perfectionnement des saints, pour l’œuvre de service, pour l’édification du corps de
Christ. Cela amena dans ma vie un énorme changement de philosophie en ce qui concerne mon concept de la raison
d'être de l’Église. Plutôt que l'évangélisation du monde, j'ai compris que la raison d'être de l'Église était le
perfectionnement des saints, c'est-à-dire de rendre les croyants forts, de les amener à la maturité, de les nourrir, de
les aimer, de les fortifier, pour qu’ils soient capables de s’engager dans l’œuvre de service, car j'ai réalisé que Dieu
nous a tous appelés et placés dans son corps et il a un plan et une raison d'être pour chacun de nous. Paul disait
que les types d’hommes décrits dans \ibiblechvs{Eph}(4:) étaient destinés au perfectionnement des saints, à l'œuvre du
service, à l'édification du corps de Christ, jusqu’à ce que nous soyons tous parvenus à l’unité de la foi et de la
connaissance du Fils de Dieu, à l’état d’homme complètement mature, fait à la mesure de la stature parfaite du
Christ; jusqu’à ce que nous ne soyons plus des enfants, flottants et entraînés à tout vent de doctrine, mais qu’en
disant la vérité avec amour nous puissions croître à tous égards en celui qui en est la tête, Christ.

Ainsi, en ayant changé ma philosophie, j’ai cessé de prêcher des sermons d'évangélisation à proprement parler, mais
j’ai commencé à enseigner la Parole de Dieu de façon régulière en vue de faire grandir les croyants.

Quand j’ai débuté dans le ministère, mes sermons étaient tous des sermons à thèmes, centrés sur l’évangélisation. Je
disposais de deux années de sermons, et donc tous les deux ans je demandais à mon supérieur hiérarchique de me
changer d’église et je déménageais dans un autre endroit pour y prêcher de nouveau mes deux années de sermons.
Je fis ceci dans quatre communautés jusqu’à ce que je me retrouve à Huntington Beach en Californie. Ma fille aînée
venait alors de commencer l’école et personnellement j’aimais vivre à Huntington Beach. C’était une jolie petite station
balnéaire qui ne comptait alors que \numprint{6000}~habitants, et j'ai commencé à vraiment bien les connaître et les aimer. Mais
je me suis bientôt retrouvé à court de sermons, parce que quand on prêche des sermons à thème, il est plutôt difficile
de trouver le bon texte. Quand vous cherchez dans toute la Bible un texte sur lequel prêcher chaque semaine, c’est
difficile parce que la Bible est un livre de taille respectable. Pourtant, chaque semaine je me retrouvais à parcourir la
Bible jusqu’à ce qu'un passage m’interpelle. Bien sûr il me fallait préparer trois sermons par semaine et il me devint
difficile de trouver mon texte et ce d’autant plus qu’il fallait qu’il se prête à l’évangélisation. Quand j’avais trouvé un
texte, j’étais capable de le développer, mais en trouver un était toujours un problème.

C’est alors que je suis tombé sur un livre intitulé \emph{L’Apôtre Jean} par Griffith Thomas, livre au milieu duquel il avait
développé un plan d’étude sur la Première Épître de Jean. Je me suis mis à lire ses plans d’études de I Jean et j'ai
trouvé que c’étaient des plans remarquables pour enseigner cette petite épître. Il y avait 43~études et je me suis dit\frcolon
\og Dis donc! Tu peux rester un an de plus à Huntington Beach rien qu’en enseignant I~Jean. \fg{} J'ai donc annoncé le
dimanche matin, que le dimanche suivant nous commencerions une étude de la Première Épître de Jean.

La première chose que Griffith Thomas expliquait dans son livre était la raison pour laquelle Jean avait écrit son
épître en premier lieu. Au premier chapitre, il dit \og Et je vous écris ceci afin que votre joie soit complète\fg{}. Au deuxième
chapitre, il dit\frcolon\og Je vous écris ceci afin que vous ne péchiez pas \fg{} et au cinquième chapitre, il ajoute \og cela je vous l’ai
écrit afin que vous sachiez que vous avez la vie éternelle\fg{}.

En annonçant à l'assemblée que nous allions commencer une étude de I Jean, j'ai ajouté\frcolon\og Attention, il y a trois
raisons pour lesquelles Jean a écrit cette petite épître. Dimanche prochain je veux que vous soyez en mesure de me
donner ces trois raisons. Quand je vous saluerai à l’entrée de l’église, si je vous demande les trois raisons pour
lesquelles Jean a écrit cette épître, je m’attends à ce que vous soyez capables de me les donner\fg{}. Des gens
m’appelèrent en milieu de la semaine pour me dire\frcolon\og Ça fait sept fois que nous relisons le texte en entier et nous
n'avons pu trouver que deux raisons. Êtes-vous sûr qu’il y en ait trois?\fg{} Je leur répondait\frcolon\og Je suis certain qu’il y en a
trois, continuez à lire.\fg{} Mon sermon ce dimanche matin là traita du but du livre. J’avais trois points\frcolon lire le livre vous
donnera le plénitude de joie, la liberté vis-à-vis du péché et l’assurance de votre salut.

Il y a six endroits où Jean dépeint Jésus comme notre exemple. Aussi ce dimanche je leur ai demandé\frcolon\og Attention,
pour la semaine prochaine je veux que vous trouviez les six endroits où Jean dépeint Jésus comme notre exemple et
les mots-clés sont \og comme il\fg{}, \og comme lui \fg{}, ou \og tel il \fg{}. Six endroits où il a dépeint Jésus comme notre exemple,
trouvez-les !\fg{}

Les gens se sont remis à lire tout le livre et il leur a fallu le faire 8, 9, 10~fois de suite pour identifier les six passages\frcolon
si nous marchons dans la lumière ; comme il est lui-même dans la lumière ; nous sommes en communion les uns
avec les autres ; si nous déclarons demeurer en lui, alors nous devons marcher comme lui a marché ; Il est notre
exemple pour vivre . Nous devons marcher comme Il marche, marcher dans la lumière comme Il est dans la lumière,
notre exemple de justice et de pureté, car nous sommes purs comme lui est pur, nous sommes justes comme lui est
juste. Il a dit que nous devons aimer comme Il nous l’a demandé. Finalement, tel Il est lui, tels nous devons être
aussi dans ce monde.

Le sermon suivant traitait des affirmations mensongères souvent faites par les gens. La Première Épître de Jean fait la
liste de sept affirmations mensongères avec les mots-clés \og si un homme dit \fg{}, \og celui qui prétend \fg{}, \og si nous disons \fg{}. J'ai
demandé à l’assemblée d'identifier ces fausses affirmations. Et la congrégation relisait le livre une fois de plus. La
semaine suivante, nous avons parlé du verbe \og connaître/savoir \fg{}. Comment savons-nous que nous savons ? Je les
faisais relire le livre. J'ai alors commencé une étude \og verset après verset \fg{} du livre dans son intégralité. Commençant
au verset un du chapitre un
et en progressant régulièrement jusqu'à la fin de I~Jean, j'ai passé une
année entière à étudier le livre.

La chose intéressante, c'était qu’en l’espace d’un an, l’église avait doublé en nombre. Je n’avais pas donné
d’invitations à recevoir Christ à chaque culte, mais nous avons eu plus de conversions et de baptêmes que pendant
aucune autre des années précédentes. La chose enthousiasmante était que les gens avaient une plus grande joie
dans leur vie avec le Seigneur que jamais auparavant. Ils faisaient l’expérience d’une réelle puissance sur le péché et
ils avaient l’assurance de leur salut.

Le prophète Ésaïe a déclaré\frcolon\og Comme la pluie et la neige descendent des cieux et n’y retournent pas sans avoir
arrosé, fécondé la terre et fait germer les plantes, sans avoir donné de la semence au semeur et du pain à celui qui
mange. Ainsi en est-il de ma parole qui sort de ma bouche\fg{} déclare le Seigneur \og . \og Elle ne retourne pas à moi sans
effet, sans avoir exécuté ma volonté et accompli avec succès ce pourquoi je l’ai envoyée.\fg{} Si Dieu nous a envoyé
cette petite épître de I Jean pour nous amener la plénitude de joie pour nous amener la libération du péché et pour
nous amener l’assurance du salut, c’est exactement ce qui va arriver aux gens quand vous leur enseignez ce livre. La
Parole de Dieu ne retourne pas sans effet. Nos paroles probablement mais Sa Parole, Non ! Si vous êtes fidèles dans
l’enseignement de Sa Parole, elle accomplira l’effet pour lequel Dieu l’a envoyée. C’est pourquoi quand vous lisez une
épître, il est toujours bon de vous demander \og Quel est le but de cette épître ? Pourquoi a-t-elle été écrite ? » Trouvez
le but et vous trouverez ce que Dieu veut accomplir dans votre vie et ce à quoi vous pouvez vous attendre quand
vous étudiez sérieusement cette épître ou cet évangile.

Il me fut possible de rester une autre année à Huntington Beach ce qui me plut beaucoup et de plus, avec la nouvelle
croissance de l’église, c’était encore plus formidable. En finissant la Première Épître de Jean, je commençais à
développer mon propre style d’enseignement textuel, verset après verset. Je me demandais \og Quel autre livre de la
Bible pourrais-je ``attaquer'' de la même manière que I Jean? \fg{} Au séminaire, un de mes professeurs nous disait que
l'étude de La Lettre aux Romains devait révolutionner n’importe quelle église. J’avais toujours entendu dire que ce
livre était fabuleux mais, je dois le confesser, je l’avais lu plusieurs fois et il ne m’avait pas vraiment emballé. Mais
j’avais très confiance en ce professeur et s’il avait dit que ce livre devait révolutionner n’importe quelle église, je
pensais que ce serait formidable de participer à une révolution. Aussi quand nous sommes arrivés à la fin de notre
étude de I Jean, j’ai annoncé à l’église\frcolon\og Dimanche prochain, nous allons commencer une étude de La Lettre de Paul
aux Romains.\fg{}

J'ai fait l’achat de tous les commentaires sur l’épître aux Romains que j'ai pu trouver et je me suis à développer des
plans d’étude similaires à ceux que j’avais utilisés pour I Jean. J'ai passé deux ans à étudier La Lettre aux Romains le
dimanche matin; l’église doubla encore de taille ; nous avions plus de gens sauvés et plus de gens baptisés que
jamais auparavant. C’était formidable; c'était passionnant.

J'ai acquis un exemplaire du Manuel Biblique de Halley en format de poche. En fait, j'ai pris l’habitude de donner ce
livre à chaque nouveau croyant. J’ai toujours dit qu'en dehors de la Bible, le premier livre que vous devez avoir dans
votre bibliothèque, c'est le Manuel Biblique de Halley. Il est rempli de bonnes informations valables sur le contexte, la
culture, l’archéologie, l’histoire. Pour un petit livre, il renferme plus de trésors et plus d'informations qu’aucun autre
livre que je connaisse. Une nouvelle édition est sortie avec une couverture qui annonçait\frcolon\og La page la plus importante
de ce livre est la page~867.\fg{} J'avais tant d'admiration pour M. Halley, que je me suis aussitôt demandé\frcolon\og Quelle page
considère-t-il comme la plus importante de ce livre ?\fg{} J'ai donc été à la page~867 où il était dit\frcolon\og Toute église devrait
avoir une méthode pour encourager la congrégation à lire systématiquement toute la Bible et, idéalement, le sermon
du dimanche devrait se baser sur un passage extrait des textes lus pendant la semaine.\fg{} Il suggérait un plan de
lecture pour lire toute la Bible en un an. Estimant que c’était un peu trop ardu, j'ai pensé que nous pourrions faire
cette lecture en deux ans. En lisant dix chapitres par semaine, quinze en abordant les Psaumes, nous pouvions
boucler la Bible en deux ans. L’idée me vint alors qu’il est possible de rester dans une église le reste de votre vie, si
vous vous mettez à enseigner toute la Bible.

J'ai ainsi découvert qu’il était beaucoup plus facile d’écrire des sermons quand j’étais limité à une petite partie de la
Bible pour y trouver mon texte; les sermons était de bien meilleure qualité, car je pouvais consacrer beaucoup plus de
temps à étudier le texte à partir duquel j’allais prêcher que quand je butinais de ci, de là dans toute la Bible. Quand
vous devez trouver votre texte dans une certaine portion des Écritures, cela vous pousse à vous consacrer à des
études sérieuses et approfondies. J’ai donc adopté la suggestion de M. Halley de mener les gens dans l'étude de la
Bible du début à la fin, et c’est devenu mon habitude depuis lors.

À l’heure actuelle (1989), c'est la septième fois que nous parcourons toute la Bible avec notre congrégation, à Calvary
Chapel Costa Mesa. J’ai considérablement ralenti. Je n’étudie plus que deux ou trois chapitres par semaine, j’ai donc
vraiment ralenti mon rythme. Cette fois encore, j’aime cette approche plus que jamais, parce qu'elle me permet
d'apprendre toujours davantage progressivement. La dernière fois que j’ai fini un parcours de toute la Bible, j'avais
déjà ralenti au rythme de cinq chapitres par semaine. Lorsque j’aurai terminé le parcours systématique en cours,
nous aurons un commentaire très complet de l’intégralité de la Bible parce que j’ai pris l’habitude de lire un nouveau
commentaire chaque fois que j’étudie la Bible du début à la fin, quelquefois deux ou trois nouveaux commentaires et
en conséquence j’ai pu lire la plupart des commentaires bibliques importants.

La très importante leçon que j’ai apprise, c’est que la meilleure façon d’apprendre est d’enseigner. Une fois que vous
commencez à enseigner, vous commencez aussi vraiment à apprendre parce qu’il vous faut absorber tellement plus
d’informations que vous ne pouvez en restituer . Il vous faut assimiler et faire un tri. Il vous faut probablement
assimiler dix fois plus que vous ne restituez. C’est donc une façon excellente d’apprendre - commencez à enseigner.

Dans la Lettre aux Hébreux, au chapitre 6, verset~\ibiblevs{He}(6:1), l’auteur écrit\frcolon\og C’est pourquoi, laissant l’enseignement
élémentaire de la parole de Christ, tendons vers la perfection, sans poser de nouveau le fondement\frcolon repentance des
œuvres mortes, foi en Dieu, doctrine des baptêmes, imposition des mains, résurrection des morts et jugement
éternel.\fg{} Avec le recul, j'ai pu faire le bilan de mes années de ministère, et comparer les 17~premières années où
j'avais \og bataillé \fg{} aux 25~années suivantes où j'ai \og surfé \fg{} dans le ministère; je \og bataillais \fg{} quand je m'efforçais d’être un
évangéliste et prêchais des sermons à thème. Il y a eu une transition bien marquée. J’ai réellement commencé à
enseigner et à me sentir à l’aise dans la 14\up{e}~année de mon ministère.

Je ne sais pas si l’Épître aux Romains a révolutionné l’église, mais il m'a vraiment révolutionné! Je n'ai plus jamais été
le même. J'ai découvert une nouvelle relation avec le Seigneur de toute première qualité! Cela a révolutionné toute
mon expérience spirituelle. Dieu m’a vraiment renversé et retourné. J’ai aussi pris conscience d'une vérité importante
en étudiant la lettre aux Romains\frcolon en se fortifiant et en mûrissant dans la Parole de Dieu, les gens sont devenus des
témoins plus efficaces pour Jésus-Christ. Christ était devenu leur vie. Nous n'avions plus besoin d'organiser des
soirées de porte-à-porte ou des programmes de témoignage. Le témoignage devint une extension naturelle de leur
vie chrétienne. Être un témoin ne consiste pas à faire quelque chose, c'est quelque chose que l'on est. Quand vous
avez mûri en Christ, la maturité de votre vie spirituelle est un témoignage auprès des autres.

Quand j'essayais d’être un évangéliste, j’ai découvert que la chose la plus frustrante dans tout le ministère, c’était de
voir le Seigneur me mettre à cœur un sermon d’évangélisation dynamique et de n’avoir aucun pécheur à qui le
prêcher dans l’église. J’étais emballé par quelques uns des sermons que le Seigneur me donnait. Des sermons
d’évangélisation fantastiques! Ils étaient vraiment si puissants dans leur logique qu’aucun pécheur n’aurait pu les
entendre sans accepter Jésus. J'allais à l’église, le coeur débordant de ce message dynamique que le Seigneur m’avait
donné. J'avais du mal à attendre le moment où j’allais le présenter. J'avais du mal à attendre le moment de l’invitation
pour voir tous les pécheurs de la salle à genoux, car j’étais sûr que ce serait le cas.

Mais, souvent j’arrivais à l’église avec ce genre de sermons me brûlant le coeur, je m’asseyais sur l’estrade pendant la
louange, je jetais un coup d'œil sur la congrégation et réalisais que je les connaissais tous par leur prénom. Pas un
seul pécheur dans la salle! Énervé, j’ajoutais quelques commentaires à mon sermon\frcolon\og Vous n’êtes que des bons à
rien! Dieu en a assez, vous ne témoignez jamais pour lui. Si vous étiez vraiment les gens que Dieu désire, vos amis
seraient ici, avec vous ce soir, vous auriez amené vos voisins pécheurs écouter la Parole de Dieu ! \fg{}

Je rejetais la responsabilité sur les saints parce que j’étais en colère qu'aucun pécheur ne soit présent. Les \og chers
saints bénis \fg{}, recevant mes reproches comme autant de coups de fouet sur le dos, se faisaient touts petits sur leurs
bancs pendant que la condamnation leur tombait dessus de tout son poids. Au lieu d’inviter les gens à accepter Christ,
je leur demandais combien de personnes voulaient consacrer leurs vies à être vraiment les témoins que le Seigneur
attendait, parce que j’avais la conviction que si les gens ne s’avancent pas pour prier devant l’autel votre sermon n’est
pas un succès.

Le problème, cependant, ne venait pas d’un manque de désir d’être de meilleurs témoins. Ils désiraient servir le
Seigneur. Le problème c’est qu’ils ne savaient pas exactement comment parce qu’on ne leur avait pas appris. Tout ce
qu’on leur avait donné, c’était le biberon. Tout ce qu’ils avaient entendu c’était \og repentez vous de vos péchés \fg{} et
\og Jésus est mort pour nous sauver de nos péchés. \fg{} Tout ce qu’ils avaient eu, c’était de l’évangélisation. Ils n’avaient
jamais été instruits dans la Parole de Dieu pour pouvoir mûrir et grandir.

Cependant, quand les saints furent équipés pour l’oeuvre de service, ils commencèrent à servir. Ils commencèrent à
amener leurs amis. L’évangélisation était le fruit d’une église forte et mature. Une église qui est solide dans la Parole
sera automatiquement une église qui évangélise. Des brebis en bonne santé vont se reproduire naturellement. C’est
tout naturel. Vous n’avez même pas à leur apprendre comment faire. Quand vous assurez la santé des brebis en leur
donnant un bon régime alimentaire, un régime régulier qui développe la croissance et la force, elles vont
naturellement se reproduire.

J’ai aussi découvert qu’en enseignant un livre de la Bible du début à la fin, vous évitez de ne parler que de vos
dadas. Il y a certains sujets dans la Bible que je trouve plus fascinants que d’autres. Il y a des sujets que j’aime
prêcher. Les sujets que je n’aime pas prêcher, je trouve le moyen de les éviter. Quand vous enseignez un livre de la
Bible du début à la fin, vous ne pouvez pas escamoter les sujets impopulaires rarement abordés et que les gens ont
besoin d’entendre. Dieu ne les aurait pas inclus dans la Parole si ce n’était pas des sujets importants. Si vous
enseignez un livre de façon continue et systématique, vous déclarerez tout le conseil de Dieu et vous mettrez l’accent
sur les points que la Bible met en valeur. Je crois qu’en étudiant la Bible vous découvrirez qu'elle l'accent sur ce que
Dieu a fait pour l’homme; C'est Dieu qui initie et c'est l'homme qui répond. Car c’est l’amour du Christ qui nous étreint
(\ibibleverse{IICo}(5:14)). Dieu a initié notre relation par son grand amour pour moi et je ne fais que répondre à cet
amour.

En reconsidérant mes anciens sermons à thème, j'ai pu alors me rendre compte que je mettais toujours l’accent sur
ce que l’homme devait faire pour Dieu. C’étaient des sermons sur la vie du croyant ; comment nous devions prier
davantage ; comment nous devions donner davantage ; comment nous devions témoigner davantage; comment nous
devions louer Dieu davantage. C’était toujours sur ce que nous devions faire pour Dieu. Mais c’est frustrant surtout
pour la congrégation. Oui, je sais que je devrais faire toutes ces choses pour Dieu mais je ne sais pas comment.
Voyez vous, si vous ne prenez que ces textes qui généralement ne sont pas au début d’un chapitre, mais qui un peu
plus loin disent \og je vous exhorte, donc, frères par les compassions de Dieu à offrir vos corps comme un sacrifice
vivant.\fg{}, et que vous n’êtes pas revenu en arrière à cette grâce de Dieu en laquelle nous demeurons fermement, sans
savoir tout ce que Dieu a fait pour nous, alors votre engagement peut être le simple fruit d’une émotion passagère. Je
suis appelé à offrir mon corps mais sans bonne raison.

Dans les Écritures, les exhortations à s’engager commencent généralement par \og donc\fg{} ou \og en conséquence\fg{}. Ces
mots ne sont jamais le commencement d’une pensée mais au contraire, ce sont des mots qui demandent une réponse
aux déclarations ou aux arguments qui les précèdent.
Paul n’a pas commencé l’Épître aux Romains par le chapitre~12, il a commencé par la chapitre~1. Il y a une
progression naturelle de la pensée tout au long de l’épître jusqu’à ce que vous arriviez au chapitre~12 où, parce que
Dieu vous a appelé, justifié et glorifié, je vous exhorte, donc, à Lui offrir vos corps.

Regardez Ephésiens, Paul commence le premier chapitre en disant\frcolon\og Béni soit le Dieu et Père de notre Seigneur
Jésus-Christ, qui nous a bénis de toutes bénédictions spirituelles dans les lieux célestes en Christ.\fg{} Dieu nous a bénis
et Paul passe trois chapitres à nous parler de toutes ces bénédictions spirituelles que nous avons en Christ. Ce n’est
pas avant d’arriver au chapitre~4 que de nouveau il utilise les mots \og en conséquence\fg{}, \og Voyez ce que Dieu a fait pour
vous, en conséquence, marchez d’une manière digne de la vocation qui vous a été adressée.\fg{} Ce n’est pas avant
d’arriver au chapitre~5 que Paul commence à vous exhorter sur la façon dont vous devez vous comporter avec votre
famille, votre femme, vos serviteurs, vos employés, mais je le répète, seulement après nous avoir donné la base de ce
que Dieu a déjà fait pour nous. Si nous n’insistons que sur ce que les gens doivent faire pour Dieu, ce n’est pas une
insistance réellement biblique.

Selon moi, la Bible nous enseigne que Dieu est l’initiateur. \ibibleverse{Jn}(3:16) déclare\frcolon\og Car Dieu a tant aimé le monde qu’Il a
donné son Fils unique\frcolon\fg{} Dieu a initié son amour envers moi. Dieu a tendu sa main vers moi. Dieu a initié ma relation
avec Lui. Il m’a choisi en Christ avant la fondation du monde. Dieu a initié toute l’affaire. Ce que je suis donc appelé
à faire, c’est répondre à Dieu. Quand vous enseignez dans cette perspective solide et biblique, vous allez découvrir
que lorsque les gens commencent réellement à comprendre Dieu et ce que Dieu a fait pour eux, ils veulent répondre
à Dieu. Vous n’aurez pas besoin de les supplier de se porter volontaires pour aider, ils se porteront volontaires d’eux-
mêmes. Vous n’aurez pas besoin d’avoir recours à des tas d'astuces pour les faire donner. Ils voudront donner. Ils
voudront répondre à Dieu. Quand ils comprendront réellement qui est Dieu et ce qu’Il a fait pour eux, ils répondront à
Dieu.

J’ai assisté à des cultes où l'on encourageait les gens à \og louer le Seigneur\fg{}, de façon à ce que Dieu les bénisse parce
qu’on leur avait dit que le Seigneur siège au milieu des louanges de son peuple. Dans ce cas, vous dites que l’homme
est l’initiateur, que vous pouvez mettre les choses en route entre vous et le Seigneur. Tout ce qu'il vous suffit de faire,
c’est de le louer un petit peu. Il va répondre et se mettre à vous bénir. La véritable louange n’est pas quelque chose
que l’on fait avec dans son coeur la motivation d’obtenir une bénédiction. Si je le loue le Seigneur dans le seul but
d’obtenir une bénédiction, ce n’est pas la véritable louange. C’est une attitude égoïste. Dans ce cas, l’objet de la
louange c’est moi, non pas Dieu. La véritable louange, c’est cette réponse automatique de mon coeur qui reconnaît la
grâce de Dieu manifestée à mon égard lorsque Dieu vient de faire quelque chose de fantastique pour moi, bien que
ma vie soit un échec lamentable. Dieu me comble d'une riche bénédiction et mon coeur répond, \og Oh Dieu, Tu es trop
grand; je n’arrive pas à croire à ton amour et à ta bonté.\fg{} C’est la forme de louange la plus pure, celle qui vient
spontanément de mon coeur quand je reconnais la grâce de Dieu dans ma vie. Je ne loue pas le Seigneur pour
pouvoir créer une atmosphère dans laquelle Dieu va venir me bénir. Mes louanges sont une réponse aux bénédictions
dont Dieu m’a comblé. Dieu est Celui qui initie. L’homme est celui qui répond.

La première Épître de Pierre commence par un remerciement à Dieu \og qui nous a régénérés par la résurrection de
Jésus-Christ d’entre les morts, pour une espérance vivante, pour un héritage qui ne peut ni se corrompre, ni se
souiller, ni se flétrir et qui vous est réservé dans les cieux, à vous qui êtes gardés en la puissance de Dieu.\fg{} Tout vient
de Dieu. Nous n’y sommes pour rien. Il parle de ce que Dieu fait. Merci à Dieu qui nous a fait naître de nouveau. Où
entrons-nous en jeu? Pierre dit\frcolon\og nous sommes gardés par la puissance de Dieu, par la foi.\fg{} C’est ici où nous entrons
en jeu, simplement en croyant que Dieu a fait tout cela pour nous. Dans \ibibleverse{Jn}(6:29) Jésus a dit\frcolon\og Ce qui est l’oeuvre
de Dieu, c’est que vous croyiez en celui qu’Il a envoyé.\fg{} Oui, la réponse humaine est importante, mais je dois savoir
ce à quoi je réponds. Il me faut connaître Dieu et savoir ce que Dieu a fait. Une personne recevra ce savoir
naturellement si vous enseignez toute la Bible pas à pas, les livres de la Bible les uns après les autres.

Donc, essentiellement, la philosophie de Calvary Chapel consiste à perfectionner les saints en vue de l’œuvre de
service et de l’édification du corps du Christ, les instruisant dans la Parole jusqu’à ce qu’ils parviennent à l’unité de la
foi et de la connaissance du Fils de Dieu à l’état d’homme fait à la mesure de la stature parfaite du Christ.

Quand vous considérez l’Église de Jésus-Christ, vous constatez qu’elle recouvre un très large éventail de styles. Quand
vous considérez notre société, vous voyez qu’il existe aussi un très large éventail de gens avec des goûts très
nombreux et variés. Ainsi d’un côté de ce large éventail, on trouve l’église liturgique très formaliste\frcolon livre de prières,
chasubles, chorales de chants psalmodiés, encens, cierges, moments désignés pour se lever, s’agenouiller, s’asseoir et
répondre. Tout est prévu à l’avance pour vous. C’est une forme d’adoration liturgique très formaliste, très rituelle. De
l’autre côté de cet éventail, il n’y a ni forme, ni programme, beaucoup de cris, beaucoup de hurlements, beaucoup de
parler en langues, des gens qui se déplacent de partout et tout le monde se levant ici ou là. Il n’y a pas d’ordre, pas
de forme; vous êtes là assis, attendant la suite des événements.

Il est vrai que certaines personnes semblent ne pouvoir communiquer avec Dieu que dans un cadre très liturgique. Ils
aiment les froissements de robes, les chorales qui psalmodient et l’odeur de l’encens et dans ce cadre ils ont le
sentiment d’adorer Dieu. Quand ils sortent du culte, ils ont le sentiment d’avoir été en présence de Dieu et ils aiment
adorer le Seigneur de cette manière. Je ne doute pas que certaines personnes expriment vraiment leur adoration et
leur amour du Seigneur dans cet environnement et communiquent avec lui dans ce cadre liturgique.

De l’autre côté, vous avez des gens qui sont très émotionnels et qui à moins d’avoir encaissé une forte décharge
d'émotions et d'avoir traversé une grande variété d'expériences physiques, n’ont pas l’impression d’avoir adoré Dieu
correctement. En fait, ils sortent souvent d’une église où l’on enseigne la Parole en disant\frcolon\og C’est le culte le plus mort
auquel j’ai jamais assisté. Je ne sais pas comment tu peux trouver quelque chose d’intéressant venant de la part de
ce vieil homme; c’était complètement mort. Pourquoi n’ont-ils pas parlé en langues ? Pourquoi n'y a-t-il pas eu de
miracles?\fg{} Tout ce qui compte pour eux, c'est une secousse émotionnelle. Ils vivent dans l’attente d’une sommet
émotionnel et dans ce sommet émotionnel, ils ont le sentiment d’adorer Dieu. C’est la façon dont ils communiquent
avec Dieu, une façon émotionnelle. Dieu sait qu’il existe des personnes du côté \og émotions \fg{}; Il sait aussi qu’il existe
des gens du côté \og liturgie \fg{}. Et Dieu les aime tous!

Parce que Dieu sait qu’il y a certaines personnes qui ne peuvent communiquer avec Lui que d’une façon liturgique, Il
dispose des églises liturgiques qui peuvent être au service des gens qui ont besoin de liturgie. Parce qu'Il sait qu’il y a
des personnes qui ne peuvent communiquer avec Lui que d’une façon très émotionnelle, Il dispose des églises
fortement émotionnelles où les gens peuvent communiquer avec Lui au travers d’expériences émotionnelles. Je
remercie Dieu pour ces églises et je vois bien leur place dans le corps de Christ. Le balancier du pendule de l’église
peut donc aller du côté hautement liturgique à une extrémité, au côté \og expérientiel \fg{} totalement non-conformiste de
l’autre.

Venant du côté liturgique, vous trouvez ces églises qui enseignent la Parole de Dieu. Leurs cultes sont une sorte de
rituel, c’est à dire que vous pouvez savoir chaque dimanche exactement ce qui va se passer. Ça se passe comme ça
depuis cent ans et vous pouvez vous sentir en sécurité parce que vous allez avoir l’appel pour l’adoration, puis le
cantique d’ouverture, les annonces, l’offrande, puis leur sermon, la bénédiction et c’est l’heure de rentrer à la maison.
Le sermon est une présentation de la Parole et il y un grand nombre de prédicateurs doués. Malheureusement,
beaucoup d’entre eux refusent de croire à l’onction et à la puissance du Saint-Esprit. Aussi, le résultat est une
orthodoxie sans vie.

Calvary Chapel croit qu'il faut enseigner la Parole de Dieu par la puissance de L’Esprit de Dieu qui change la vie du
peuple de Dieu. Si vous n’avez que l’Esprit sans la Parole et la fondation de la Parole, alors vous menez les gens à ne
connaître que des expériences qui restent superficielles. Si vous n’avez que la Parole de Dieu sans l’Esprit, alors vous
menez les gens à une orthodoxie sans vie. Il faut la puissance de l’Esprit de Dieu pour amener des changements,
mais il faut la Parole de Dieu pour donner la substance et pour établir la fondation. C’est ce mélange de la Parole de
Dieu enseignée selon la puissance de l’Esprit de Dieu qui amène le changement dans la vie des gens.

Calvary Chapel reconnaît le besoin de la puissance de l’Esprit, mais nous reconnaissons aussi le besoin d’une solide
fondation et de l’enseignement de la Parole. Pour enseigner la Parole de façon efficace, il faut en effet l’onction et la
puissance du Saint Esprit pour que la personne qui enseigne la Parole de Dieu exerce de façon habituelle les dons de
parole de sagesse, de parole de connaissance et de prophétie, ainsi ces dons sont à l'œuvre dans la vie du pasteur
alors même qu’il enseigne la Parole. Voilà où Calvary Chapel se situe dans l’éventail de l’Eglise.

Parce que notre société a radicalement changé dans les trente cinq dernières années il est nécessaire de diviser
l’éventail de l’Eglise avec une ligne perpendiculaire indiquant vers le haut \og Haute Vision du Monde\fg{}, et vers le bas
\og Basse Vision du Monde\fg{}. La Haute Vision du Monde correspond à des gens qui sont hautement structurés,
hautement organisés et ont des programmes hautement élaborés. Tout est à sa place dans sa petite case pour bien
s'intégrer à un ensemble construit de façon très soigneuse. La Basse Vision du Monde, c’est l’attitude \og relax \fg{},
décontractée, de ceux qui prennent les choses comme elles viennent.


\begin{figure}[h!]
\begin{picture}(250,200)
\thicklines
\put(125,35){\line(0,1){130}}
\put(20,100){\line(1,0){210}}
\put(125,190){\makebox(0,0)[b]{\textsc{Haute vision du monde}}}
\put(125,170){\makebox(0,0)[b]{\textsc{(Formaliste / Liturgique)}}}
\put(125,30){\makebox(0,0)[t]{\textsc{Basse vision du monde}}}
\put(125,10){\makebox(0,0)[t]{\textsc{(Informel / Relax)}}}
\put(10,100){\makebox(0,0)[r]{\rotatebox{90}{\textsc{Dépendants}}}}
\put(240,100){\makebox(0,0)[l]{\rotatebox{-90}{\textsc{Indépendants}}}}
\put(75,140){\makebox(0,0){\small{Majorit\'e des \'eglises}}}
\put(75,125){\makebox(0,0){\small{10\% de la pop.}}}
\put(175,75){\makebox(0,0){\small{Calvary Chapel}}}
\put(175,60){\makebox(0,0){\small{90\% de la pop.}}}
\end{picture}
\end{figure}


Des deux côtés de la Haute et de la Basse Visions du Monde, il y a ceux qui sont dépendants et ceux qui sont
indépendants. Ceux qui sont dépendants ont besoin de s’appuyer sur quelque chose ou sur quelqu'un. Ils ont besoin
d’une église qui mette l’accent sur leur dépendance vis à vis de l’église et sur la dépendance de l’église vis-à-vis d’eux.
Vous avez ceux qui sont hautement organisés mais indépendants, ceux qui sont peu structurés et dépendants et ceux
qui sont relax et indépendants. La majorité des églises aujourd’hui appartiendraient plutôt à la catégorie \og dépendants,
hautement organisés \fg{} du genre \og tout le monde fait partie d’un comité \fg{} et \og tout le monde connaît ses responsabilités \fg{}.
Ce genre d’églises dit\frcolon\og Nous dépendons de vous.\fg{} Nous dépendons de vos dons, de votre présence et vous
dépendez de nous pour votre vie spirituelle et votre salut. Quand vous n’êtes pas présent au culte, une personne
désignée à l’avance vous appelle le lendemain pour voir si tout va bien et pour savoir pourquoi vous n’êtes pas venu.
Vous n’oseriez pas aller visiter une autre église car on vous accuserait de quitter le Seigneur. Sans toujours vous le
dire, ils pensent que votre salut dépend de votre fidélité à cette église.

Calvary Chapel, de l’autre côté appartient à la catégorie \og décontracté, relax, indépendant \fg{}. Nous attirons les gens qui
sont plus décontractés et plus indépendants, les gens qui n’ont généralement pas besoin de s’appuyer sur quelqu’un
et qui ne souhaitent pas qu’on s’appuie sur eux. Ils peuvent porter des tee shirts et pas de cravates ou des costumes
trois pièces s’ils le veulent, personne ne se soucie de la façon dont vous vous habillez.

Les structures sociales ayant beaucoup changé aux États Unis ces dernières années avec une très nette évolution vers
une société hautement technologique, il semble que 90\% de la population ait adopté un mode de vie plutôt
indépendant et décontracté, notamment ici en Californie du Sud, alors que les 10\% restants correspondent au modèle
dépendant et organisé. Il en ressort que Calvary Chapel et quelques autres églises similaires pêchent tous seuls dans
90\% de l'étang. Les autres 90\% des églises viennent nous voir et se demandent, \og Qu’est ce qu’ils font ?\fg{} Ils étudient
notre église, en s’efforçant de comprendre quel est notre programme et disent \og Ah voilà, c’est parce qu’ils laissent les
jeunes marcher pieds nus. Voilà leur secret. \fg{} Et ils trouvent toutes sortes de \og secrets\fg{} qui expliquent pourquoi les
gens sont attirés et viennent remplir les églises Calvary Chapel.

Ce qu’ils ne réalisent pas, c’est que c’est l’Esprit de Dieu à l’œuvre par la Parole de Dieu dans la vie du Peuple de Dieu
qui est le vrai secret ! et non de suivre ou de se conformer à l’église traditionnelle. Les gens ne se sentent pas
menacés. Ils n’ont pas le sentiment qu’on va les coincer ou leur donner un manuel de l’École du Dimanche et leur
dire\frcolon\og Oh, Merci à Dieu, Frère, cela fait trois semaines que vous venez, nous avons maintenant besoin de vous pour
enseigner à l’École du Dimanche.\fg{} On ne va pas vous forcer à faire quoique ce soit ; votre service va être quelque
chose qui va venir de l'intérieur de vous parce que vous répondez au Seigneur.

La philosophie de Calvary Chapel c’est \og donner et servir \fg{} plutôt que \og prendre et être servi \fg{}. Vous allez découvrir
que beaucoup de ministères existent dans le but d’être servis ! Et ils ne se gênent pas pour vous le faire savoir. \og Nous
avons besoin de votre soutien pour faire vivre ce ministère. Ce ministère dépend de vous.\fg{} J’en suis venu à la
conclusion que tout ministère qui dépend des hommes pour son existence et son fonctionnement devrait mourir, et la
meilleure chose que nous puissions faire est de laisser mourir. Calvary Chapel existe donc pour servir et nous mettons
l’accent sur donner\frcolon donner aux gens; servir les gens.

Il y avait un homme très riche qui était vice-président d’une compagnie d’outillage au Texas et qui travaillait aussi
dans l’industrie pétrolière. Il fréquentait Calvary Chapel de façon assez régulière et nous étions devenus amis avec lui
et sa femme. Cependant, pendant tout ce temps, il ne cessait de dire à sa femme \og Quand vont-ils nous réclamer de
l’argent ?\fg{} Il s’attendait à tout moment qu’on lui demande de l'argent.

Le dimanche d’avant la fête de Thanksgiving, j’ai annoncé que nous avions bien des raisons d’être reconnaissants et
combien Dieu avait été bon avec nous cette année-là. J’ai ajouté \og Cependant, malheureusement, certains passent par
des difficultés et n’ont pas beaucoup de raisons d’être pleins de gratitude.\fg{} Alors que je commençais à parler des
problèmes et des difficultés financières de certains, l’homme donna un coup de coude à sa femme en disant\frcolon
\og Finalement, il y arrive. Je savais bien que ce petit discours finirait par arriver un jour ou l’autre.\fg{}

Cependant, je n'ai pas terminé mes propos de la façon qu’il attendait. J'ai dit, \og Aussi si vous êtes dans le besoin en
cette période de Thanksgiving, si vous avez de réelles difficultés financières, voyez notre pasteur assistant après le
culte, l’église sera heureuse de vous offrir une dinde et tout ce qu’il vous faudra pour votre dîner de Thanksgiving.
Nous prions pour que vous ayez une merveilleuse fête de Thanksgiving.\fg{} L’homme fut absolument stupéfait. Nous
opérions simplement selon la vérité des Écritures, comme Jésus le disait, il y a plus de bonheur à donner qu’à
recevoir. C’est notre philosophie\frcolon donner gratuitement la Parole de Dieu aux gens, donner gratuitement de nous
même en servant les gens\frcolon aller au-delà de ce qui est demandé .

De même, le ministre de Dieu doit servir plutôt qu’être servi. Quelque part en chemin, il s’est produit une énorme
inversion dans la terminologie et l’idée du ministère. Le mot \og ministre\fg{} signifie en réalité \og serviteur\fg{}, ainsi Josué était
le ministre de Moïse. Cela signifiait qu’il était le serviteur de Moïse, il s’occupait de répondre aux besoins de Moïse. Il
était son garçon de courses. C’est ce que signifie le mot \og ministre\fg{}. Cependant, je suis étonné de voir la façon dont
certains ministres s’énervent quand quelqu’un leur demande d'être au service de la congrégation. \og Imagine toi qu’il
m’a appelé pour venir le chercher en voiture! Ne sait-il pas que je suis le ministre ici?\fg{} Si vous êtes le ministre, il est
normal que cette personne vous ai demandé de faire le taxi. Jésus a dit \og quiconque veut être le premier parmi vous
sera l’esclave de tous.\fg{}

Le ministre est un serviteur. Rappelez vous\frcolon c’est Jésus qui a pris une serviette, l'a mise autour de sa taille et s’est mis
à faire le tour de ses disciples pour leur laver les pieds. C’était le travail d’un serviteur, pas celui du maître. Sentiers
poussiéreux, sandales ouvertes, les pieds étaient toujours sales, aussi quand quelqu’un vous rendait visite le plus
humble des serviteurs de la maison avait la tâche de venir à la porte enlever les sandales de l’invité et de laver ses
pieds dans une bassine d’eau. C’était le rôle que Jésus a choisi et illustré par son exemple lors de la Sainte Cène,
Jésus a dit à ses disciples\frcolon\og Comprenez vous ce que je vous ai fait? Si donc je vous ai lavé les pieds, moi le Seigneur
et le Maître, vous aussi, vous devez vous laver les pieds les uns les autres.\fg{}

En d’autres termes, l’idée c'est que nous devons être des serviteurs et nous devons penser qu'être dans le ministère
c'est être un serviteur. Le livre \og Le Style Jésus\fg{} de Gayle Erwin devrait vous faire découvrir tout ce que la notion de
service chrétien recouvre réellement et ce que le ministère devrait être. L’église entière, en commençant par le
pasteur doit être là pour le service des autres. Nous ne sommes pas là pour que les autres soient à notre service.
Nous n’attendons pas que les gens nous servent, nous recherchons des façons de les servir.

Devant le large éventail que représente tout le corps de Christ, la philosophie de Calvary Chapel est de remplir la
petite partie de l’éventail que Dieu nous a appelés à remplir et d’être fidèles à cette vocation. Nous nous efforçons de
ne pas perdre de vue tout le Corps de Christ et la raison d'être de tout le Corps. Ainsi, dans le corps de Christ, les
seuls avec qui nous pouvons nous trouver en conflit sont de ceux qui ne mènent pas les gens à une relation
personnelle avec Jésus-Christ. Cela peut paraître bizarre à certains mais malheureusement, il existe des églises qui en
sont arrivées au point où elles ne mènent plus les gens à vivre une relation personnelle avec Jésus-Christ.

Nous ne faisons pas concurrence aux églises qui conduisent les gens à Jésus-Christ; nous ne les combattons pas.
Nous ne sommes pas là pour les combattre; nous sommes là pour combattre le diable et proclamer Jésus-Christ.
Jésus a dit à ses disciples \og vous recevrez une puissance, celle du Saint-Esprit survenant sur vous, et vous serez mes
témoins à Jérusalem, dans toute la Judée, dans la Samarie et jusqu’aux extrémités de la terre.\fg{} Leur témoignage pour
Christ devait commencer à Jérusalem et il fut très efficace à Jérusalem. Quelques mois après la naissance de l’église,
les disciples comparaissaient devant des tribunaux pour répondre aux accusations suivantes\frcolon\og Vous avez rempli la
ville entière avec la doctrine de Jésus-Christ.\fg{} C’est ce que j’appelle une église qui a réussi ! Plaise à Dieu que nous
soyons amenés devant les tribunaux, accusés d’avoir rempli toute la ville de la doctrine de Jésus-Christ. Je dirai\frcolon
\og Louons le Seigneur! \fg{}

Les persécutions avaient dispersé l’Église de Jésus-Christ à travers toute la Judée et partout où ils allaient, les
chrétiens prêchaient Jésus-Christ. Nous lisons que Philippe se rendit alors en Samarie et prêcha Christ aux Samaritains
; beaucoup de Samaritains crurent et furent baptisés quand ils virent les miracles qu’accomplissait Philippe. Puis nous
lisons que le Saint Esprit dit\frcolon\og Mettez moi à part Barnabas et Paul pour l’œuvre à laquelle je les ai appelés.\fg{} Après
avoir jeûné et prié, ils leur imposèrent les mains et Paul et Barnabas s’embarquèrent pour Chypre. Plus tard, Paul
amena l’Évangile en Asie Mineure, à Rome, en Grèce et en Macédoine. Thomas emmena l’Évangile en Inde. Trente
ans seulement après la naissance de l’Église, Paul écrivit à l’église de Colosse\frcolon\og Cet évangile est parvenu chez vous,
tout comme il porte des fruits et fait des progrès dans le monde entier». En seulement trente ans, les disciples
avaient répandu le message dans le monde entier.

Quand nous avons commencé à Calvary Chapel en~1965 avec seulement 25~personnes, j’ai décidé de faire de ces 25~personnes,
 les personnes qui connaîtraient le mieux la Parole de Dieu dans toute la région du port. J’ai commence à
enseigner cinq nuits par semaine\frcolon deux nuits à l'église, trois nuits dans des études bibliques chez les particuliers. Nous
avons adopté le deuxième chapitre du livre des Actes comme notre modèle; \og ils persévéraient dans l’enseignement
des apôtres, dans la communion fraternelle, dans la fraction du pain et dans les prières.\fg{} Et aussi nous avons décidé
que ces activités constitueraient les éléments essentiels de notre culte et de notre communion. L’accent serait
l’enseignement de la Parole, \og l’enseignement des apôtres\fg{}. Nous leur enseignerions une doctrine solide. Nous les
enseignerions sur Dieu. Nous les enseignerions sur Jésus-Christ. Nous les enseignerions sur le Saint Esprit. Nous les
enseignerions sur l’homme. Nous les enseignerions sur le péché. Nous les enseignerions sur le salut et nous les
enseignerions sur le retour de Jésus-Christ. Doctrine solide! L’enseignement des Apôtres.

Nous avons commencé à développer la communion fraternelle, la \og koinonia\fg{}; nous sommes vraiment devenus une
\og unité intégrée\fg{} et nous avons commencé à prendre soin les uns des autres au sens physique et au sens spirituel,
priant les uns pour les autres, unissant nos vies dans la prière, nous aidant les uns les autres au sens physique. Si une
personne du groupe avait un besoin précis nous allions tous l’aider, ce qui créait une communion fraternelle forte.
Nous nous réunissions pour des études bibliques et pour rompre le pain ensemble.

Dans le livre des Actes, il est dit que pendant qu'ils faisaient ces choses, le Seigneur ajoutait chaque jour à l’Eglise
ceux qui étaient sauvés. Ayant commencé à enseigner les gens, cette communion fraternelle se mit à se transformer
en une union, une unité, une participation commune dans la prière et l’amour et dans le soutien, et alors que nous
commencions à rompre le pain ensemble, à adorer le Seigneur ensemble, à nous souvenir de Jésus qui est mort pour
nous et alors que nous commencions à prier ensemble, le groupe se mit à grandir. Ma femme dirigeait un groupe de
prières pour femmes dans le quartier durant la semaine et j’en dirigeais un autre pour les hommes le samedi soir.
Nous avions aussi un groupe d’hommes que nous avions nommés comme anciens qui allait rendre visite aux malades
et prier avec eux. Comme nous commencions à assurer fidèlement ces activités, nous avons pu voir que le Seigneur
commençait à ajouter chaque jour à l’église ceux qui étaient sauvés.

Au bout de six mois, nous étions arrivés à 50~personnes. Au bout d’un an, nous étions une centaine. Dix-huit mois
plus tard, nous étions à la recherche d’un autre local, la petite église où nous nous réunissions étant devenue trop
petite. Une église luthérienne nous avait promis ses locaux, car ils construisaient un nouveau bâtiment. Mais leur
projet prenant du retard, nous avons dû commencer à nous réunir chez eux le dimanche après-midi en attendant
patiemment que nous puissions devenir propriétaire des lieux. Nous avons attendu deux ans, mais notre croissance
faisait que nous remplissions maintenant l'église luthérienne. Si bien que lorsque nous avons enfin pu emménager
dans l'église luthérienne, elle était déjà trop petite Nous avons donc été forcés d'emménager dans la \og Petite Chapelle\fg{}
que nous avions construite. Nous y sommes restés deux ans jusqu’à ce qu'elle soit pleine à craquer et nous nous
sommes alors installés sous un chapiteau de cirque.

Pendant que nous construisions notre nouvelle église, nous avons dû en agrandir les plans à trois reprises. Nous
croissions si vite que l’architecte ne pouvait pas dessiner ses plans assez vite pour s'adapter à notre expansion. Nous
avions dépassé la capacité de l’église à trois reprises pendant qu’on en dessinait les plans et quand nous avons ouvert
les portes, nous avons commencé avec deux cultes le dimanche matin mais au bout de deux semaines, il nous a fallu
passer à trois cultes par dimanche.

Pendant que nous grandissions et nous répandions dans notre Jérusalem, nous avons commencé à nous étendre en
Judée. Mon fils, Chuck Junior, Greg Laurie, Jeff Johnson, Mike MacIntosh, Raul Ries, Jon Courson, Don McLure et
Steve Mays ont établi des groupes d’études bibliques et des communautés dans toute la Californie. Puis de Judée
nous nous sommes passés en Samarie avec d’autres églises Calvary Chapel solides apparaissant dans les états de
Washington, Oregon, Floride, Kansas, New York, Pennsylvanie, Arizona, Nouveau-Mexique et beaucoup d’autres.

Maintenant nous sommes arrivés aux extrémités de la terre\frcolon l’Angleterre, la Hongrie, la Yougoslavie, la
Tchécoslovaquie, l’Allemagne, la Suisse, le Japon, Singapour, Hong-Kong, Taiwan, les Philippines, la Thaïlande, l’Inde,
l’Égypte, l’Ouganda, le Pérou, le Chili, le San Salvador, le Guatemala et beaucoup d’autres pays.

Si le Seigneur tarde, continuerons-nous à voir ce type de croissance exponentielle, cette explosion? Cela peut arriver
si seulement nous restons flexibles et laissons l’Esprit nous guider, l’Esprit agir; n’essayez pas de trop organiser les
choses. Laissez Dieu s’en charger. Contentez vous d’enseigner la Parole, amenez les gens à une relation d’amour avec
Jésus-Christ et les autres et célébrez la communion avec eux.

Quand Dieu a établi la nation d’Israël, la forme de gouvernement était une théocratie, c’est à dire le peuple était régi
par Dieu. Ils ne devraient pas être comme les autres nations avec un roi à leur tête. Ils devraient être une nation qui
serait différente car gouvernée par Dieu. Ce fut un bien triste jour dans leur histoire que le jour où ils vinrent trouver
Samuel pour lui dire, \og Nous voulons que tu nommes un roi qui règne sur nous comme les autres nations.\fg{} Dans cette
théocratie, Dieu avait établi la nation d’Israël mais il avait appelé Moïse à en être le responsable terrestre et Dieu par
l’intermédiaire de Moïse guidait le peuple. Quand les choses devinrent trop lourdes à gérer pour Moïse, les
responsabilités trop grandes, il rassembla soixante dix des anciens d’Israël représentant les douze tribus et l’Esprit de
Dieu qui était sur Moïse descendit sur eux aussi. Ils se mirent à gouverner avec Moïse.

Il y avait cependant des fois où les gens amenaient un problème qu’ils n’arrivaient pas à résoudre à l’un des anciens.
Dans ce cas, le problème était présenté à Moïse, et Moïse à son tour, allait devant Dieu et Dieu donnait à Moïse la
solution du problème. En sens inverse, Moïse passait la réponse aux anciens qui la passaient au peuple.

Sous l’autorité de Moïse, on trouvait aussi, Aaron et la prêtrise de la tribu de Lévi qui supervisaient les aspects
spirituels de la nation. Ainsi pendant que les anciens s’occupaient des disputes et des différends légaux et
économiques du peuple, les prêtres supervisaient les aspects spirituels de la vie du peuple.

Ce qui suit représente un diagramme du type de gouvernement que Dieu avait instauré dans sa nation, Israël et un
diagramme de ce que Calvary Chapel considère être la contrepartie néo-testamentaire du gouvernement de Dieu pour
l’Eglise.

\begin{figure}[h!]
\begin{tabular}{p{0.45\textwidth}|p{0.45\textwidth}}
Ancien Testament & Nouveau Testament\\
Gouv.\ théocratique & Gouv.\ de l’Église\\
\hline
\hline
Dieu & Jésus\\
\hline
Moïse & Pasteur\\
\hline
Juges, prêtres, assistants & Anciens \& diacres, conseil d’église, pasteurs\\
\hline
Enfants d’Israël & Congrégation\\
\end{tabular}
\end{figure}

Nous croyons que c’est la forme de gouvernement que Dieu désire pour son Église. Jésus-Christ est la tête du corps,
l’Église ; Il a établi l’episkopos ou évêque, que nous appelons le pasteur, qui est redevable devant Jésus et qui doit
accepter et endosser la responsabilité de guider et diriger le ministère de l’Eglise locale, guidé directement par Jésus-
Christ. Sous le pasteur, dans certains cas, vous avez des pasteurs assistants, équivalent des prêtres, sous la gestion
de Moïse.

Vous trouvez aussi le Conseil d’Église. Le Conseil d’Église discute et décide des affaires de l’église, des dépenses des
fonds de l’église, des demandes d’aide formulées par différents groupes missionnaires et les différents ministères.

La réunion du Conseil devrait toujours commencer par la prière. Lorsqu’il faut se décider par vote, vous devriez prier
avant de voter. Vous devriez demander au Seigneur de vous montrer ce qu’Il veut que vous fassiez. En toutes choses
il faut être guidé, dirigé par le Seigneur. Les pasteurs assistants supervisent des aspects variés de l’église au sens
spirituel\frcolon collégiens, adultes célibataires, couples mariés, groupes d’intérêt particulier. Quand ils rencontrent un
problème qu’ils ne savent pas résoudre, ils devraient rechercher le conseil du pasteur principal qui comme eux devrait
chercher conseil auprès du Seigneur.

Si quelqu’un dans l’église parle à un membre du Conseil de quelque chose que l’église devrait faire à son avis, ce
projet est présenté à la réunion du Conseil. Nous en discutons et nous prions ensemble, et souvent le Conseil va me
dire\frcolon\og Chuck, que devons-nous faire à ton avis?\fg{} Le conseil reconnaît que Dieu m’a appelé à être le pasteur de
l’église, le berger.

Quand des questions se posent à notre réunion de conseil, invariablement, avant que la décision ne soit prise, le
conseil me demande mon avis parce qu’ils respectent le fait que Dieu m’a appelé et a fait grandir ce ministère et
qu'ainsi il m’a utilisé comme Son instrument. Mais souvent je réponds\frcolon\og Mes amis, je n’ai vraiment pas d’avis sur la
question ; prions et cherchons la volonté du Seigneur.\fg{} Je les laisse alors aller de l’avant et prendre la décision sans
aucun avis de ma part.

D’autres fois, j’ai un avis bien tranché et je l’exprime\frcolon\og Je crois que le Seigneur veut que nous fassions ceci. J’ai prié à
ce sujet et j’ai vraiment le sentiment que c’est ce que le Seigneur veut nous voir faire. Invariablement, parce que ces
hommes reconnaissent l’onction de Dieu sur ma vie, le vote ira dans mon sens. Je suis honnête et franc avec ces
hommes, je n’essaye pas de les manipuler; je n’essaye pas de faire mon \og one-man-show \fg{}. Nous sommes ouverts
dans nos discussions et dans les situations que nous rencontrons et ils respectent l’intégrité et l’autorité que le
Seigneur a placé sur ma vie. Mais sans question, le Seigneur est définitivement la tête du corps de l’église. Je ne suis
qu’un serviteur qui exécute ses ordres. À Calvary Chapel, le pasteur n’est pas un mercenaire à la solde de ceux qui
l'ont embauché. Il y a beaucoup d’églises dans lesquelles le pasteur est un employé. Il est embauché par le conseil et
peut être congédié par le conseil. Il devient un employé et finit par faire les quatre volontés des membres du conseil
qui eux en réalité, gouvernent l’église. Mais souvent ces hommes sont des hommes d’affaires et ne sont pas les
hommes les plus spirituels dans l’église. Dans ce cas, l’église finit par être gouvernée par des hommes plutôt que
gouvernée par Jésus-Christ.

Il existe cependant des dangers attenants à la forme de gouvernement théocratique, principalement parce qu’il y a
certains pasteurs qui désobéissent à ce que le Seigneur a dit\frcolon\og quiconque veut être chef doit être le serviteur de
tous.\fg{} Il y a des pasteurs qui ont abusé de leurs pouvoirs. Ils n’ont pas une comptabilité nette devant le Conseil
concernant les aspects financiers de l’église. Ils ne recherchent pas l’avis et les conseils du Conseil avant de prendre
les décisions importantes qui concernent le fonctionnement de l’église. Ils essaient de faire leur \og one-man-show \fg{}!

Il est important d’avoir un Conseil d’Église mais tout aussi important de ne pas le mettre en place trop rapidement.
Quand une nouvelle œuvre démarre, la Bible dit ne pas imposer les mains sur un homme avec précipitation.
Connaissez bien les hommes. À chaque fois que nous recherchons des nouveaux membres du Conseil, je cherche
toujours des hommes qui prient avec moi depuis des années dans notre groupe de prières des hommes du samedi
soir. Je peux leur faire confiance. Je sais que ce sont des hommes de prière, des hommes qui chercheront les conseils
et la direction de Dieu, de la même façon que je recherche moi même la direction de Dieu, des hommes qui ont été
fidèles à la réunion de prière du samedi soir avec moi.

J’ai mentionné le fait qu’il est important de ne pas nommer un Conseil trop rapidement. L’exemple suivant illustre bien
les raisons de ce Conseil. L’homme qui est responsable de la communauté coréenne de notre église est un médecin. Il
ne reçoit aucun salaire pour son ministre auprès des coréens. Il gagne sa vie en tant que pédiatre et allergologue. La
communauté coréenne devenant très nombreuse, ils se sont dit\frcolon\og Il nous faut mettre en place un conseil pour la
communauté coréenne. \fg{} Alors cet homme nomma des membres du conseil et me demanda de venir au culte pour
imposer les mains sur les hommes qu’il avait choisis et je le fis. La semaine même où nous avions imposé les mains
sur ces hommes, prié pour eux et nommés au conseil, ils se réunirent et demandèrent au pasteur de démissionner. Ils
disaient\frcolon\og Soit vous abandonnez votre pratique médicale, soit vous démissionnez de votre pastorat. Nous estimons
qu’il nous faut un pasteur à temps plein et votre pratique médicale vous détourne de votre ministère.\fg{} L’homme était
complètement dévasté. Il ne savait pas ce qu’il devait faire. Aussi me demanda-t-il ce que j’en pensais. Je lui dis de
dissoudre le Conseil\frcolon\og Dieu vous a appelé à être le pasteur de cette communauté; le Conseil ne vous a pas appelé à
en être le pasteur. Laissez-les partir.\fg{} Ainsi, nous les avons ordonnés puis défroqués la semaine d'après! Ce n’est
qu’un des problèmes auxquels vous vous exposez si vous n’avez pas l’habitude de prier ensemble et si vous ne
connaissez pas vraiment les hommes qui servent avec vous sur le Conseil.

D’un autre côté, pour votre protection, il vous faut un Conseil constitué d’hommes sur qui vous pouvez compter, parce
qu’il y a des décisions qui doivent être prises qui ne vont pas être du goût de tout le monde, des décisions qui vont
créer des divisions dans l’église si c'est vous qui les prenez personnellement. Il y a plusieurs années alors que j’étais
pasteur à Tucson en Arizona, nous organisions, chaque année, un pique-nique sur le Mont Lemon pour la fête
nationale du 4~juillet. Il y avait un très beau parc public avec des terrains de base-ball, de football, etc... C’est là que
nous allions chaque année pour jouer à des jeux de ballon et pique-niquer. C'était toujours un grand moment de
communion fraternelle pour l'église.

Un type du genre \og hyper, super spirituel\fg{} venait d'arriver à l'église, et un groupe l’avait suivi chez nous. Il possédait
un grand terrain au sommet du Mont Lemon et il avait décidé qu'il serait formidable que le pique-nique se passe sur
son terrain. Le problème, c'est qu'il n’y avait ni toilettes ni eau courante. Qu'importe! il suggérait que l’on passe toute
la journée à prier. Ne valait-il pas mieux passer la journée à prier et à méditer devant le Seigneur plutôt
que se livrer à des activités aussi frivoles que des jeux de ballon. Ainsi donc cet homme avait persuadé une partie de
la congrégation qu'il fallait célébrer le 4~juillet de façon spirituelle. Nous irions tous prier sur sa propriété.

D'autres personnes disaient, par contre\frcolon\og Si vous allez sur son terrain, nous ne venons pas. Nous n’allons pas imposer
à nos enfants de passer la journée à un endroit sans toilettes, si vous allez là, nous ne venons pas. Ce à quoi le
groupe \og super spirituel \fg{} répliqua\frcolon\og Si vous allez au grand parc, nous n’irons pas. !Nous n’allons pas exposer nos
enfants à la racaille pendant le week-end.\fg{}

Tout le monde vint me trouver en me demandant \og OK Chuck qu’est ce qu’on va faire?\fg{} C’était une situation sans
issue. Quel que soit mon choix, j'étais sûr de me faire des ennemis. Je répondis\frcolon\og Eh bien, prions pour que le
Seigneur nous guide, et lors de la réunion du Conseil nous prendrons une décision.\fg{}

Pendant la réunion qui suivit, le Conseil déclara\frcolon\og C’est bête d’aller à l’endroit non équipé, nous ne pouvons pas
emmener 150~personnes dans un endroit sans toilettes; nous irons au grand parc public.\fg{} C’est donc le Conseil qui
décida que nous irions au grand parc public. Il se trouvait que j’étais du même avis, que je pensais que c’était la
décision la plus sage, mais techniquement, c’est le Conseil qui avait pris la décision.

Quand j'ai annoncé que le Conseil avait décidé d’aller au grand parc public, ces gens super spirituels m’ont appelé
tout en émoi. Je leur ai dit\frcolon\og Vous savez, ce serait vraiment formidable, n’est ce pas, d’avoir une journée de prière,
nous devrions bientôt en organiser une, mais le Conseil a pris sa décision.\fg{} Vous voyez, je pouvais encore être à leur
service . Ils ne s'étaient pas polarisés contre moi, ils s'étaient polarisés contre le Conseil.

Ainsi, le Conseil est là comme une protection pour le pasteur, un fusible entre vous et les gens quand des décisions
difficiles sont prises qui ne sont pas toujours du goût de tout le monde. Votre Conseil peut vous éviter d’être en
rupture avec votre congrégation et vous permet de continuer à exercer un ministère auprès d’eux. Il a une fonction
très importante et chaque église doit, je crois, dès qu’elle a des homme qualifiés, nommer un Conseil pour superviser
les opérations et les dépenses et pour prendre les décisions qui doivent être prises.

En conclusion, je crois que Calvary Chapel a une compréhension à la fois biblique et équilibrée de l'église, de sa
fonction dans le monde, et de sa dépendance totale par rapport à l’Esprit Saint de Dieu pour la guider et la diriger
vers le succès quand elle proclame fidèlement la Bonne Nouvelle de la Croix de Jésus-Christ et l’espérance du Salut en
Lui seulement.

Il y a un verset bien précis qui me vient à l’esprit quand je réfléchis au miracle de Calvary Chapel\frcolon\og Or à Celui qui par
la puissance qui agit en nous, peut faire infiniment au-delà de tout ce que nous demandons ou pensons, à Lui la
gloire dans l’Église et en Christ Jésus, dans toutes les générations aux siècles des siècles. Amen.\fg{} (\ibibleverse{Eph}(3:20))
